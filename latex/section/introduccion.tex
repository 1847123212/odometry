\documentclass[main.tex]{subfiles}

\begin{document}
% no \IEEEPARstart

En los sistemas de navegación es necesario tener métodos
de estimación de posición y velocidad a partir de ciertos
sensores innerciales o encoders.

Se necesita contar con una biblioteca que permita hacer estas
estimaciones que sea utilizable en cualquier microcontrolador
(${\mu}C$) siendo ligera, rápida y haciendo uso eficiente de la
memoria de datos.

\subsection{Definición de odometría y Dead-Reckoning}
\textbf{Odometría} es el estudio de la predicción de la posición
relativa de un móvil mediante el uso de ruedas las cuales
tienen discos encoder.respecto a un punto de referencia mediante
ecuaciones matemáticas ya establecidas.

\textbf{Dead-Reckoning(DR)} es el procedimiento matemático basada
en ecuaciones trigonométricas para la estimación de la posición
y velocidad.

Adicionalmente es necesario implementar filtros para poder mejorar
la estiación, estos suelen ser los filtros de partículas y los
filtros de Kalman extendido \cite{DR:kalman_missing}.

\subsection{Bibliotecas para operaciones matriciales y vectoriales}
Existen otras bibliotecas para operaciones matriciales y vectoriales
escritas C++. Dependiendo del tipo de operación a realizar, están
optimizadas. Las tres más reconocidas son:

\begin{itemize}
\item Armadillo
\item IT++
\item Newmat
\item Eigen
\end{itemize}

Entre las mencionadas, Armadillo es la más eficiente y rápida
\cite{LibSpeedweb:Armadillo}-\cite{LibCompare} pero no ninguna
cumple con todos los requisitos indicados.

\subsection{Problema de usar bibliotecas de terceros}
Existen varios criterios por los que se decide crear una nueva
biblioteca que haga este tipo de operaciones matriciales,
vectoriales y otros:

\begin{itemize}
\item Todos los microcontroladores presentan compilador para C
      pero un buen número no cuenta con compiladores en C++.
\item Las bibliotecas de terceros cuentan con muchas más cosas
      de las que se necesitan y que nunca serán usadas por lo
      que consumen más memoria de lo necesario.
\item Ninguna cuenta con todos los objetos matemáticos que se
      necesitan trabajar a la vez.
\item Generan código binario de más por ser de propósito general.
\end{itemize}

\subsection{Recodo - Biblioteca en C}
Dados los problemas mencionados, se opta por implementar una
biblioteca que en C que sea de propósito específico a la cual
llamaremos Recodo, por Dead-\textbf{Rec}koning y
\textbf{Odo}metría. De esta forma podemos tener control
completo sobre la administración de la memoria.

\hfill



% An example of a floating figure using the graphicx package.
% Note that \label must occur AFTER (or within) \caption.
% For figures, \caption should occur after the \includegraphics.
% Note that IEEEtran v1.7 and later has special internal code that
% is designed to preserve the operation of \label within \caption
% even when the captionsoff option is in effect. However, because
% of issues like this, it may be the safest practice to put all your
% \label just after \caption rather than within \caption{}.
%
% Reminder: the "draftcls" or "draftclsnofoot", not "draft", class
% option should be used if it is desired that the figures are to be
% displayed while in draft mode.
%
%\begin{figure}[!t]
%\centering
%\includegraphics[width=2.5in]{myfigure}
% where an .eps filename suffix will be assumed under latex,
% and a .pdf suffix will be assumed for pdflatex; or what has been declared
% via \DeclareGraphicsExtensions.
%\caption{Simulation Results}
%\label{fig_sim}
%\end{figure}

% Note that IEEE typically puts floats only at the top, even when this
% results in a large percentage of a column being occupied by floats.


% An example of a double column floating figure using two subfigures.
% (The subfig.sty package must be loaded for this to work.)
% The subfigure \label commands are set within each subfloat command, the
% \label for the overall figure must come after \caption.
% \hfil must be used as a separator to get equal spacing.
% The subfigure.sty package works much the same way, except \subfigure is
% used instead of \subfloat.
%
%\begin{figure*}[!t]
%\centerline{\subfloat[Case I]\includegraphics[width=2.5in]{subfigcase1}%
%\label{fig_first_case}}
%\hfil
%\subfloat[Case II]{\includegraphics[width=2.5in]{subfigcase2}%
%\label{fig_second_case}}}
%\caption{Simulation results}
%\label{fig_sim}
%\end{figure*}
%
% Note that often IEEE papers with subfigures do not employ subfigure
% captions (using the optional argument to \subfloat), but instead will
% reference/describe all of them (a), (b), etc., within the main caption.


% An example of a floating table. Note that, for IEEE style tables, the
% \caption command should come BEFORE the table. Table text will default to
% \footnotesize as IEEE normally uses this smaller font for tables.
% The \label must come after \caption as always.
%
%\begin{table}[!t]
%% increase table row spacing, adjust to taste
%\renewcommand{\arraystretch}{1.3}
% if using array.sty, it might be a good idea to tweak the value of
% \extrarowheight as needed to properly center the text within the cells
%\caption{An Example of a Table}
%\label{table_example}
%\centering
%% Some packages, such as MDW tools, offer better commands for making tables
%% than the plain LaTeX2e tabular which is used here.
%\begin{tabular}{|c||c|}
%\hline
%One & Two\\
%\hline
%Three & Four\\
%\hline
%\end{tabular}
%\end{table}


% Note that IEEE does not put floats in the very first column - or typically
% anywhere on the first page for that matter. Also, in-text middle ("here")
% positioning is not used. Most IEEE journals/conferences use top floats
% exclusively. Note that, LaTeX2e, unlike IEEE journals/conferences, places
% footnotes above bottom floats. This can be corrected via the \fnbelowfloat
% command of the stfloats package.

\end{document}

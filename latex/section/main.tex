\documentclass[conference]{IEEEtran}

\usepackage[utf8]{inputenc}
\usepackage{subfiles}
\usepackage{mathtools}
\usepackage{amssymb}

% *** GRAPHICS RELATED PACKAGES ***
\ifCLASSINFOpdf
  \usepackage[pdftex]{graphicx}
  % declare the path(s) where your graphic files are
  % \graphicspath{{../pdf/}{../jpeg/}}
  % and their extensions so you won't have to specify these with
  % every instance of \includegraphics
  % \DeclareGraphicsExtensions{.pdf,.jpeg,.png}
\else
  % or other class option (dvipsone, dvipdf, if not using dvips). graphicx
  % will default to the driver specified in the system graphics.cfg if no
  % driver is specified.
  % \usepackage[dvips]{graphicx}
  % declare the path(s) where your graphic files are
  % \graphicspath{{../eps/}}
  % and their extensions so you won't have to specify these with
  % every instance of \includegraphics
  % \DeclareGraphicsExtensions{.eps}
\fi

% correct bad hyphenation here
\hyphenation{op-tical net-works semi-conduc-tor}


\begin{document}
%
% paper title
% can use linebreaks \\ within to get better formatting as desired
\title{
  Recodo: biblioteca en C optimizada para cálculos de
  Odometría y Dead-reckoning para la estimación de estados
  de una antena seguidora de un drone.
}


% author names and affiliations
% use a multiple column layout for up to three different
% affiliations
\author{
  \IEEEauthorblockN{Anthony Fluker Cueva}
  \IEEEauthorblockA{Universidad Nacional de Ingeniería\\
    Lima, PERU\\
    Email: zhudkencio@gmail.com}
  \and
  \IEEEauthorblockN{Frank A. Moreno Vera}
  \IEEEauthorblockA{Hack-Zeit Research\\
    Lima, PERU\\
    Email: frank.moreno@hackzeit.com}
}

% use for special paper notices
%\IEEEspecialpapernotice{(Invited Paper)}


% make the title area
\maketitle

% IEEEtran.cls defaults to using nonbold math in the Abstract.
% This preserves the distinction between vectors and scalars. However,
% if the conference you are submitting to favors bold math in the abstract,
% then you can use LaTeX's standard command \boldmath at the very start
% of the abstract to achieve this. Many IEEE journals/conferences frown on
% math in the abstract anyway.

% no keywords

\begin{abstract}
%\boldmath
Este artículo presenta la creación de una biblioteca en C para cálculos
de odometría, creada y optimizada para ser aplicada en la estimación de
la posición de un drone con procesadores de bajos recursos como lo son
los Arduino además de compararla con otras bibliotecas existentes. Los
elementos matemáticos son descritos y estructurados computacionalmente.
La conversión entre estos tipos de datos es mostrada. Se explica también
la forma de uso en las diversas partes de un sistema implementado sobre
drones y se muestran conclusiones.
\end{abstract}


% For peer review papers, you can put extra information on the cover
% page as needed:
% \ifCLASSOPTIONpeerreview
% \begin{center} \bfseries EDICS Category: 3-BBND \end{center}
% \fi
%
% For peerreview papers, this IEEEtran command inserts a page break and
% creates the second title. It will be ignored for other modes.
\IEEEpeerreviewmaketitle


\section{Introducción}
\subfile{introduccion}

\section{Objetos matemáticos a tratar}
\subfile{objetos_matematicos}

\section{Implementación de los objetos matemáticos}
\subfile{implementacion}

\section{Conversión entre objetos matemáticos}
\subfile{conversion}

\section{Pruebas de testeo}
\subfile{testeo}

\section{Conclusiones}
\subfile{conclusiones}

\begin{thebibliography}{1}

\bibitem{IEEEhowto:kopka}
H.~Kopka and P.~W. Daly, \emph{A Guide to \LaTeX}, 3rd~ed.\hskip 1em plus
0.5em minus 0.4em\relax Harlow, England: Addison-Wesley, 1999.

\bibitem{DR:kalman_missing}
Haitao Bao; Wai-Choong Wong, "An indoor dead-reckoning algorithm with map matching," Wireless Communications and Mobile Computing Conference (IWCMC), 2013 9th International , vol., no., pp.1534,1539, 1-5 July 2013
doi: 10.1109/IWCMC.2013.6583784

\bibitem{LibSpeedweb:Armadillo}
C. Sanderson. "Armadillo C++ linear algebra library - Speed". [online]
Web oficial. Disponible en: http://arma.sourceforge.net/speed.html

\bibitem{LibSpeedpdf:Armadillo}
C. Sanderson. "Armadillo: An Open Source C++ Linear Algebra Library for Fast
Prototyping and Computationally Intensive Experiments".[online] Technical
Report, NICTA, 2010. Disponible en:
http://arma.sourceforge.net/armadillo\_nicta\_2010.pdf

\bibitem{LibCompare}
"OpenCV vs. Armadillo vs. Eigen vs. more! Round 3: pseudoinverse test".
[online] Disponible en: http://nghiaho.com/?p=1726

\end{thebibliography}

\end{document}
